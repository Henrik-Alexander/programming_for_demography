\usepackage[margin=1in]{geometry}
\newcommand*{\authorfont}{\fontfamily{phv}\selectfont}
\usepackage[]{mathpazo}
\usepackage{abstract}
\renewcommand{\abstractname}{}    % clear the title
\renewcommand{\absnamepos}{empty} % originally center
\newcommand{\blankline}{\quad\pagebreak[2]}

\providecommand{\tightlist}{%
  \setlength{\itemsep}{0pt}\setlength{\parskip}{0pt}} 
\usepackage{longtable,booktabs}

\usepackage{parskip}
\usepackage{titlesec}
\titlespacing\section{0pt}{12pt plus 4pt minus 2pt}{6pt plus 2pt minus 2pt}
\titlespacing\subsection{0pt}{12pt plus 4pt minus 2pt}{6pt plus 2pt minus 2pt}

\titleformat*{\subsubsection}{\normalsize\itshape}

\usepackage{titling}
\setlength{\droptitle}{-.25cm}

%\setlength{\parindent}{0pt}
%\setlength{\parskip}{6pt plus 2pt minus 1pt}
%\setlength{\emergencystretch}{3em}  % prevent overfull lines 

\usepackage[T1]{fontenc}
\usepackage[utf8]{inputenc}

\usepackage{fancyhdr}
\pagestyle{fancy}
\usepackage{lastpage}
\renewcommand{\headrulewidth}{0.3pt}
\renewcommand{\footrulewidth}{0.0pt} 
\lhead{}
\chead{}
\rhead{\footnotesize POSC 0000: A Class with an R Markdown
Syllabus -- Fall 2016}
\lfoot{}
\cfoot{\small \thepage/\pageref*{LastPage}}
\rfoot{}

\fancypagestyle{firststyle}
{
\renewcommand{\headrulewidth}{0pt}%
   \fancyhf{}
   \fancyfoot[C]{\small \thepage/\pageref*{LastPage}}
}

%\def\labelitemi{--}
%\usepackage{enumitem}
%\setitemize[0]{leftmargin=25pt}
%\setenumerate[0]{leftmargin=25pt}




\makeatletter
\@ifpackageloaded{hyperref}{}{%
\ifxetex
  \usepackage[setpagesize=false, % page size defined by xetex
              unicode=false, % unicode breaks when used with xetex
              xetex]{hyperref}
\else
  \usepackage[unicode=true]{hyperref}
\fi
}
\@ifpackageloaded{color}{
    \PassOptionsToPackage{usenames,dvipsnames}{color}
}{%
    \usepackage[usenames,dvipsnames]{color}
}
\makeatother
\hypersetup{breaklinks=true,
            bookmarks=true,
            pdfauthor={ ()},
             pdfkeywords = {},  
            pdftitle={POSC 0000: A Class with an R Markdown Syllabus},
            colorlinks=true,
            citecolor=blue,
            urlcolor=blue,
            linkcolor=magenta,
            pdfborder={0 0 0}}
\urlstyle{same}  % don't use monospace font for urls


\setcounter{secnumdepth}{0}





\usepackage{setspace}

\title{POSC 0000: A Class with an R Markdown Syllabus}
\author{Steven V. Miller}
\date{Fall 2016}


\begin{document}  

		\maketitle
		
	
		\thispagestyle{firststyle}

%	\thispagestyle{empty}


	\noindent \begin{tabular*}{\textwidth}{ @{\extracolsep{\fill}} lr @{\extracolsep{\fill}}}


E-mail: \texttt{\href{mailto:svmille@clemson.edu}{\nolinkurl{svmille@clemson.edu}}} & Web: \href{http://svmiller.com/teaching}{\tt svmiller.com/teaching}\\
Office Hours: W 09:00-11:30 a.m.  &  Class Hours: TR 02:00-03:45 p.m.\\
Office: 230A Brackett Hall  & Class Room: \emph{online}\\
	&  \\
	\hline
	\end{tabular*}
	
\vspace{2mm}
	


\section{Dozent}\label{dozent}

Mein Name ist Henrik-Alexander Schubert, nennt mich bitte stets Henrik,
und ich bin Doktorand am Max-Planck-institut für demografische Forschung
und an der Universität Oxford. Ich habe einen Masterabschluss in
Demografie und Abschlüsse in Soziologie und Politikwissenschaft. Meine
Interessen sind quantiative und komputergestützte Methoden,
Programmieren, Sport (vor Allem Rudern und Hockey),

\section{Voraussetzungen}\label{voraussetzungen}

Es sind keine Programmiererfahrung vorausgesetzt, aber Erfahrungen in R,
Python oder anderen Programmiersprachen sind hilfreich. Eine
Bereitschaft für mindestens 4 Stunden Arbeit in der Woche (2 Stunden
Seminar + 2 Stunden Vor- und Nachbereitung) ist essentiell.
Programmieren kann Spaß machen, aber aller Anfang ist müßig. Desweiteren
ist ein Interesse an Datenanalysen, Software, Mathematik, Logik oder
Automatisierungen von Prozessen hilfreich. Technische Voraussetzungen
sind der Besitz und Zugang zu einem Computer und/oder Zugang zum
Datenlabor. Darüber hinaus sind zwei Fertigkeiten hilfreich und
entscheidend für den Arbeitsaufwand. Erstens, gute Englischkenntnisse
sind hilfreich, weil die Dokumentation, die meisten und besten Quellen
und die Diskussionsforen auf Englisch sind. Außerdem sind die englischen
Resourcen meistens zugänglicher. Zweitens, das Behrrschen der 10-Finger
Schreibtechnik am Computer ist hilfreich. Man kann erstens zeitgleich
Lesen und Schreiben, schneller und mit weniger Fehlern schreiben.
Programmieren heißt vor Allem fehlerfrei den Computer Anweisungen in
Schrift zu geben, wenn das Schreiben bereits viel Energie und
Aufmerksamkeit auf sich zieht, dann bleibt wenig für die eigentliche
Arbeit übrig.

\section{Lernziele}\label{lernziele}

Der Kurs verfolg eine Reihe von Lernzielen, die sowohl im Bereich der
Kenntnisse, der Fertigkeiten als auch der Characterbildung liegen.
Erstens und Vordergründig soll die Programmiersprache Python oder das
Statistikprogramm \href{https://www.r-project.org/}{\textbf{R}} gelernt
werden. Jedoch sind die \textbf{grundlegenden Konzepte beim
Programmieren}, wie \emph{Loops}, \emph{Funktionen} und
\emph{Datenstrukturen} auf viele andere Programmiersprachen übertragbar.
Somit werden die Konzepte stets im Vordergrund stehen. Desweiteren wird
die \textbf{Organisation von einem quantitativen Forschungsprojekt}
vermittelt. In den meisten Kursen wird der Fokus entweder auf das
Schreiben oder die Analyse gelegt, aber die Qualität und auch die Dauer
eines Projekts ist unmittelbar abhängig von strukturellen Entscheidungen
bei der Ordnerstruktur, Datenaufbereitung, dem Forschungsdesign und der
Verschriftlichung. Darüber hinaus sollen auch noch einige
\emph{Soft-skills} vermittelt werden. Erstens, eine generelle
\textbf{Problemlösungskompetenz} wird erlernt. Programmieren zwingt
einen zu einem strukturierten und logischenDenken. Somit verbessern sich
Fähigkeiten bei der Identifikation eines Problems, Entwickeln von
verschiedenen Lösungen, Abwegen von Optionen. Desweiteren wird das
\textbf{strategisches Denken} geschult. Projekte, Texte und Analysen
werden auf ein Ziel ausgerichtet und im Voraus geplant.

\section{Prüfungsleistungen}\label{pruxfcfungsleistungen}

\begin{itemize}
\tightlist
\item
  Ein kleines Forschungsprojekt in
  \href{https://www.r-project.org/}{\textbf{R}} oder
  \href{https://www.python.org/}{{[}\textbf{Python}{]}}.
\item
  wöchtenliche Hausaufgaben (ca. 2 Stunden Bearbeitungszeit, inklusive
  Wiederholung der Inhalte)
\end{itemize}

\section{Kommunikation}\label{kommunikation}

\begin{itemize}
\tightlist
\item
  inhaltliche Fragen: Werden im Forum auf Stud.IP gestellt und kollektiv
  beantwortet.
\item
  organisatorische Fragen:

  \begin{itemize}
  \tightlist
  \item
    Kursrelevant: an mich
  \item
    Prüfungsrelevant: an das Prüfungsamt
  \end{itemize}
\item
  Beschwerden:

  \begin{itemize}
  \tightlist
  \item
    falls es strukturelle Probleme des Kurses sind oder Konflikte mit
    Kommilitonen: an mich
  \item
    Konflikt mit dem Dozierenden: an die Lehrstuhlinhaberin
    Prof.~Doblhammer oder den StuRa
  \end{itemize}
\end{itemize}

\section{Hilfsmittel}\label{hilfsmittel}

\begin{itemize}
\tightlist
\item
  für konkrete Befehle die Dokumentation: help({[}Befehlsname{]})
\item
  das Internet: stackoverflow, cran-r.org, python.org, chatGPT
\item
  textbücher:
\item
  Kommilitonen
\end{itemize}

\section{Ablauf}\label{ablauf}

\begin{itemize}
\tightlist
\item
  Das Seminar ist inhaltlich aufgeteilt in 10 Themenblöcke
\item
  Die Veranstaltungen bauen aufeinander auf und sätzen die Bearbeitung
  der Hausaufgaben sowie das Verständnis aus den vorherigen Wochen
  voraus
\item
  Hausaufgaben sind in \href{https://rmarkdown.rstudio.com/}{RMarkdown}
  zu bearbeiten und einzureichen (Samt Codesegmenten)
\end{itemize}

\subsection{1. Veranstaltung: Einleitung (Literature: Wickham: Grammar
of graphics, Wickham: R for data science,
Intro)}\label{veranstaltung-einleitung-literature-wickham-grammar-of-graphics-wickham-r-for-data-science-intro}

\begin{quote}
Vorbereitung: 1. Lesen der Texte. 2. Lernen der 10-Finger Schreibtechnik
\end{quote}

\begin{itemize}
\tightlist
\item
  Installation von R und RStudio
\item
  Wie sieht ein Projektordner aus
\item
  Aufbau von Rstudio (\emph{Console, R-script, Environments})
\item
  Was ist ein Programm (\emph{Beispiel: ``Hello, World'', ``Hello,
  Student''})
\item
  RMarkdown (\emph{Open new file}, \emph{Formatting}, \emph{Write code})
\item
  Hans-Rosling Gedenkplot
\end{itemize}

\begin{quote}
\textbf{Hausaufgabe:} Erstellen Sie eine Grafik samt Code basierend auf
den Daten von der Gapminder-Foundation in einem RMarkdown-Dokument. Die
Grafik soll kurz interpretiert werden. Hilfsmittel: Ko-operation mit
Kommilitionen, Internet, und das Video .
\end{quote}

\subsection{2. Veranstaltung: Datentypen, Objekte, Subsetting und
Visualisierungen}\label{veranstaltung-datentypen-objekte-subsetting-und-visualisierungen}

\begin{quote}
Vorbereitung: 1. Lesen der Texte, 2. Watch the video on \ldots{}
\end{quote}

\begin{itemize}
\tightlist
\item
  Assignment operator: ´\textless-´ (\textasciitilde/code/12/wer.R)
\item
  Vektoren, Matrizen, Listen (\textasciitilde/code/w2/types.R,
  \textasciitilde/code/w2/containers.R)
\item
  Subsetting von Vektoren, Matrizen und Listen
\item
  base R vs.~tidyverse -\textgreater{} ``Viele Wege führen nach Rom''
\item
  Laden von Daten (\emph{Beispiel csv. Rostock Temperaturdaten von
  {[}url{]}})
\end{itemize}

\begin{quote}
\begin{quote}
\textbf{Hausaufgabe:} Laden sie die Wetterdaten für Rostock vom
deutschen Wetterdienst. 1. Erstellen sie eine Grafik vom zeitlichen
Verlauf der tagesdurchschnittstemperatur.(Liniendiagramm bzw.
Zeitreihendiagram) 2. Filtern Sie die Daten von den Sommermonaten (Juni,
Juli, August) und Wintermonaten (Dezember, Januar, Februar). 3.
Erstellen sie jeweils ein Histogram der Tagesdurchschnittstemperatur für
den Sommer und den Winter. 4. Welche Monate waren die wärmsten und
welche die kältesten Monate? 5. Wie groß ist der Unterschied (Differenz,
prozentual) der Durchschnittstemperatur zwischen Juni und Dezember?
\end{quote}
\end{quote}

\subsection{3. Veranstaltung: Loops, Conditionals und
Rechnungen}\label{veranstaltung-loops-conditionals-und-rechnungen}

\begin{quote}
Vorbereitung:
\end{quote}

\begin{itemize}
\tightlist
\item
  mathematische Operatoren: +, -, /, *, \^{}
\item
  boolean operators: \&, \textbar, ==, !=, \textgreater=, \textless=,
  \textless, \textgreater{}
\item
  if, if\_else, if \ldots{} else
\item
  Aufbau von Loops (Input, Body, Output)
\item
  Verwendungsmöglichkeiten von Loops (Wiederholung, reduziert Code)
\item
  Progressbar
\item
  Vor- und Nachteile von Loops ()
\end{itemize}

\begin{quote}
\begin{quote}
\textbf{Hausaufgabe:} Schreib eine Loop welche eine Faktorisierung der
Zahl 5 berechnet (\(!5\)). Teste im Anschluss ob die Zahl 10 ist.
\end{quote}
\end{quote}

\subsection{\texorpdfstring{4. Veranstaltung: Funktionen und Packages
(Sources: 1.
\href{https://www.youtube.com/watch?v=JP7ITIXGpHk&list=PLhQjrBD2T3817j24-GogXmWqO5Q5vYy0V&index=2}{Video-CS50})}{4. Veranstaltung: Funktionen und Packages (Sources: 1. Video-CS50)}}\label{veranstaltung-funktionen-und-packages-sources-1.-video-cs50}

\begin{quote}
Vorbereitung:
\href{https://www.youtube.com/watch?v=JP7ITIXGpHk&list=PLhQjrBD2T3817j24-GogXmWqO5Q5vYy0V&index=2}{CS50}
von 1:36 Stunde
\end{quote}

\begin{itemize}
\tightlist
\item
  Aufbau einer Funktion: input -\textgreater{} body -\textgreater{}
  output (\textasciitilde/code/w4/make\_mail.R,
  \textasciitilde/code/w4/quadrieren.R)
\item
  initial-values von Funktionen
\item
  Vor- und Nachteile von Funktionen
\end{itemize}

\begin{quote}
\begin{quote}
\textbf{Hausaufgabe:} Schreib eine Funktion namens gerade, die testet,
ob eine Zahl gerade oder ungerade ist. Wenn die Zahl ungerade ist, dann
soll die Funktion FALSE ansonsten TRUE anzeigen.
\end{quote}
\end{quote}

\subsection{5. Veranstaltung: Kausalität und Simulationen (Literature:
Telling stories with data, Kapitel 2, Kosuke Imai (2021): Introduction
to quantitative methods, Kapitel
1)}\label{veranstaltung-kausalituxe4t-und-simulationen-literature-telling-stories-with-data-kapitel-2-kosuke-imai-2021-introduction-to-quantitative-methods-kapitel-1}

\begin{quote}
Vorbereitung: Read the literature
\end{quote}

\begin{itemize}
\tightlist
\item
  potential outcome framework
\item
  individueller und durchschnittlicher kausaler Effekt
\item
  directed acyclic graphs (DAGs)
\item
  randomized-control trials
\end{itemize}

\begin{quote}
\textbf{Hausaufgabe:} Zeichne einen DAG und simuliere die Daten für den
Einfluss von Rauchen auf die Sterblichkeit.
\end{quote}

\subsection{6. Veranstaltung: Regression - Regression (Literatur:
Travor, Hastie, : Introduction to statistical learning, Kapitel
1)}\label{veranstaltung-regression---regression-literatur-travor-hastie-introduction-to-statistical-learning-kapitel-1}

\begin{quote}
Vorbereitung:
\end{quote}

\begin{itemize}
\tightlist
\item
  Geschichte der linearen Regression (Beispiel: Größe über Generationen
  hinweg)
\item
  Regressionsgleichung: \(y=\alpha + \beta + X + \epsilon\)
\item
  Was ist \(\alpha\) und \(\beta\)
\item
  Berechnung: Methode der kleinsten Quadrate und likelihood-Methode
\item
  statistische Annahmen: \(iid.\) = independent and identically
  distributed observations
\item
  Vorhersage
\end{itemize}

\begin{quote}
\begin{quote}
\textbf{Hausaufgabe:} Gompertz hat die Alterung in der Bevölkerung mit
einer log-linearen Funktion beschrieben. Die Aussage des Gesetzes ist,
dass die altersspezifischen Sterblichkeitsraten nach dem Alter 30
exponentiell steigen. Nutze die Daten von der Human Mortality Database
für Schweden und berechne eine Funktion für das Jahr 1850 und 2022.
Interpretiere die y-Achsenabschnitte und die Steigung. Wie lässt sich
die Steigung interpretieren und wie lässt sich der y-Achsenabschnitss
interpretieren. Was besagt die Änderung über die Zeit?
\end{quote}
\end{quote}

\subsection{7. Veranstaltung: Karten und Wahrscheinlichkeit (Literatur:
Kosuke Imai: Quantitative Social Science, Kapitel 6 \&
7)}\label{veranstaltung-karten-und-wahrscheinlichkeit-literatur-kosuke-imai-quantitative-social-science-kapitel-6-7}

\begin{quote}
Vorbereitung:
\end{quote}

\begin{itemize}
\tightlist
\item
  Bootstrap
\item
  Illustration vom ``Gesetz der großen Zahlen'', ``Asymptotische
  Theories'' (Würfel)
\end{itemize}

\begin{quote}
\begin{quote}
\textbf{Hausaufgabe: Erstellen sie eine Deutschlandkarte für die
Lebenserwartung auf Kreisebene aus dem Zeitschriftenaufsatz von Rau und
Schmertmann (2021).}
\end{quote}
\end{quote}

\subsection{8. Veranstaltung: Web-scraping (Literatur: Matthew Salganik:
Bit-by-Bit.)}\label{veranstaltung-web-scraping-literatur-matthew-salganik-bit-by-bit.}

\begin{quote}
Vorbereitung:
\end{quote}

\begin{itemize}
\tightlist
\item
  was ist web-scraping, API
\item
  Beispiele für Webscraping (Facebook, Wohnungsmarkt)
\item
  Download einer Tabelle von Wikipedia -\textgreater{} packages: httr2,
  rvest
\item
  Analyse der Tabelle
\end{itemize}

\begin{quote}
\textbf{Hausaufgabe:} Erstelle eine Datenbank mit allen Angestellten des
Lehrstuhls von Gabrielle Doblhammer samt Informationen zu Adresse,
Sprechzeit.
\end{quote}




\end{document}

\makeatletter
\def\@maketitle{%
  \newpage
%  \null
%  \vskip 2em%
%  \begin{center}%
  \let \footnote \thanks
    {\fontsize{18}{20}\selectfont\raggedright  \setlength{\parindent}{0pt} \@title \par}%
}
%\fi
\makeatother